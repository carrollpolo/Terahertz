\section{总结与展望/ConclusIon And RecommendAtIons}

本文首先从6G典型应用场景及其性能需求出发,分析了研究巨容量太赫兹无线通信的必要性。接着,概述了6G网络架构和太赫兹在其中的重要作用以及国内外太赫兹通信发展现状。然后,根据太赫兹信号的产生与接收方式阐述了太赫兹无线通信系统的架构与分类。针对太赫兹成像与传感领域,本文重点介绍了脉冲时域成像系统与太赫兹成像技术在糖尿病患者的检测的应用。

THz通信具备高数据传输速率和宽带宽等优点, 信息传输过程中安全性能高, 有望引入6G系统中. 通过探索THz产生新方法、发展新天线技术来提高THz信号的增益,优化系统资源分配, 进而实现小型化、低功耗和低成本的THz通信系统, 增加通信覆盖面, 提升数据传输速率和传输距离, 使6G无线网络为人们提供生活的便利。